\begin{exercise}[1.1]
    \begin{enumerate}
        \item Let $Y$ be the plane curve $y = x^2$ (i.e., $Y$ is the zero set of 
        the polynomial $f = y - x^2$). Show that $A(Y)$ is isomorphic to a polynomial ring 
        in one variable over $k$

        \item Let $Z$ be the plane curve $xy = 1$. Show that $A(Z)$ is not isomorphic to 
        a polynomial ring in one variable over $k$. 
        
        \item Let $f$ be any irreducible quadratic polynomial in $k[x, y]$, and let 
        $W$ be the conic defined by $f$. Show that $A(W)$ is isomorphic to $A(Y)$ and $A(Z)$.
        Which one is it and when? 
    \end{enumerate}
\end{exercise}

\begin{exercise}[1.2]
    \emph{The Twisted Cubic Curve.}
    Let $Y \subseteq \aa^3$ be the set $Y = \{(t,t^2,t^3) \mid t \in k\}$.
    Show that $Y$ is an affine variety of dimension 1. Find generators 
    for the ideal $I(Y)$. Show that $A(Y)$ is isomorphic to a polynomial ring 
    in one variable over $k$. We say that $Y$ is given by the \emph{parametric representation} 
    $x = t$, $y = t^2$, $z = t^3$). 
\end{exercise}

\begin{exercise}[1.3]
    Let $Y$ be the algebraic set in $\aa^3$ defined by the two polynomials
    $x^2 - yz$ and $xz - x$.
    Show that $Y$ is a union of three irreducible components.
    Describe them and find their prime ideals.
\end{exercise}

\begin{exercise}[1.4]
    If we identify $\aa^2$ with $\aa^1 \times \aa^1$ in the natural way, show
    that the Zariski topology on $\aa^2$ is not the product topology of the
    Zariski topologies on the two copies of $\aa^1$.
\end{exercise}

\begin{exercise}[1.5]
    Show that a $k$-algebra $B$ is isomorphic to the affine coordinate ring of
    some algebraic set in $\aa^n$, for some $n$, if and only if $B$ is a
    finitely generated $k$-algebra with no nilpotent elements.
\end{exercise}

\begin{exercise}[1.6]
    Any nonempty open subset of an irreducible topological space is dense and
    irreducible.
    If $Y$ is a subset of a topological space $X$, which is irreducible in its
    induced topology, then the closure $\overline{Y}$ is also irreducible.
\end{exercise}

\begin{exercise}[1.7]
    \begin{enumerate}
        \item Show that the following conditions are equivalent for a topological
        space $X$: $(i)$ $X$ is noetherian; $(ii)$ every nonempty family of
        closed subsets has a minimal element; $(iii)$ $X$ satisfies the ascending
        chain condition for open subsets; $(iv)$ every nonempty family of open
        subsets has a maximal element.
      \item A noetherian topological space is \emph{quasi-compact,} i.e., every
        open cover has a finite subcover.
      \item Any subset of a noetherian topological space is noetherian in its
        induced topology.
      \item A noetherian space which is also Hausdorff must be a finite set with
        the discrete topology.    
    \end{enumerate}
\end{exercise}

\begin{exercise}[1.8]
    Let $Y$ be an affine variety of dimension $r$ in $\aa^n$.
    Let $H$ be a hypersurface in $\aa^n$, and assume $Y \not\subseteq H$.
    Then every irreducible component of $Y \cap H$ has dimension $r-1$.
    (See $(7.1)$ for a generalization.)
\end{exercise}

\begin{exercise}[1.9]
    Let $\mathfrak{a} \subseteq A = k[x_1,\ldots,x_n]$ be an ideal which can be
    generated by $r$ elements.
    Then every irreducible component of $Z(\mathfrak{a})$ has dimension $\ge n-r$.
\end{exercise}

\begin{exercise}[1.10]
    \begin{enumerate}
        \item If $Y$ is any subset of a topological space $X$ then $\dim Y \leq \dim X$. 
        \item If $X$ is a topological space which is covered by a family of open
          subsets $\{U_{i}\}$, then $\dim X = \sup \dim U_i$.
        \item Give an example of a topological space $X$ and a dense open subset $U$
          with $\dim U < \dim X$. 
        \item If Y is a closed subset of an irreducible finite-dimensional topological space $X$, and if $\dim Y = \dim X$, then $Y= X$. 
        \item Give an example of a noetherian topological space of infinite dimension.  
    \end{enumerate}
\end{exercise}

\begin{exercise}[1.11]
    Let $Y \subseteq \aa^3$ be the curve given parametrically by
    $x = t^3$, $y= t^4$, $z = t^5$.
    Show that $I(Y)$ is a prime ideal of height $2$ in $k[x,y,z]$ which cannot
    be generated by $2$ elements.
    We say $Y$ is \emph{not a local complete
    intersection}---cf.~\emph{(Ex.~$2.17$)}. 
\end{exercise}

\begin{exercise}[1.12]
    Give an example of an irreducible polynomial $f \in \mathbf{R}[x,y]$,
    whose zero set $Z(f)$ in $\aa_{\mathbf{R}}^2$ is not irreducible
    \emph{(cf.~$1.4.2$)}.
\end{exercise}

\begin{exercise}[2.1]
    Prove the ``homogeneous Nullstellensatz,'' which says if $\fa
    \subseteq S$ is a homogeneous ideal, and if $f \in S$ is a homogeneous
    polynomial with $\deg f > 0$, such that $f(P) = 0$ for all $P \in
    Z(\fa)$ in $\bbP^n$, then $f^q \in \fa$ for some $q >
    0$.
\end{exercise}

\begin{exercise}[2.2]
    For a homogeneous ideal $\fa \subseteq S$, show that the following
    conditions are equivalent:
    \begin{itemize}
        \item[(\emph{i}.)] $Z(\fa) = \emptyset$ (the empty set);
        \item[(\emph{ii}.)] $\sqrt{\fa} =$ either $S$ or the ideal $S_+ =
        \bigoplus_{d > 0}S_d$;
        \item[(\emph{iii}.)] $\fa \supseteq S_d$ for some $d > 0$. 
    \end{itemize}
\end{exercise}

\begin{exercise}[2.3]
    \begin{enumerate}
        \item If $T_1 \subseteq T_2$ are subsets of $S^h$, then $Z(T_1) \supseteq
        Z(T_2)$.
      \item If $Y_1 \subseteq Y_2$ are subsets of $\bbP^n$, then $I(Y_1) \supseteq
        I(Y_2)$.
      \item For any two subsets $Y_1,Y_2$ of $\bbP^n$, $I(Y_1 \cup Y_2) = I(Y_1)
        \cap I(Y_2)$.
      \item If $\fa \subseteq S$ is a homogeneous ideal with
        $Z(\fa) \ne \emptyset$, then $I(Z(\fa)) =
        \sqrt{\fa}$.
      \item For any subset $Y \subseteq \bbP^n$, $Z(I(Y)) = \overline{Y}$.
    \end{enumerate}
\end{exercise}

\begin{exercise}[2.4]
    \begin{enumerate}
        \item There is a one-to-one inclusion-reversing correspondence between
        algebraic sets in $\bbP^n$ and homogeneous radical ideals of $S$ not equal
        to $S_+$ given by $Y \mapsto I(Y)$ and $\fa \mapsto
        Z(\fa)$. \emph{Note:} Since $S_{+}$ does not occur in this correspondence, 
        it is sometimes called te \emph{irrelevant} maximal ideal of $S$. 
        
        \item An algebraic set $Y \subseteq \bbP^n$ is irreducible if and only if $I(Y)$ is a prime ideal. 
        \item Show that $\bbP^n$ itself is irreducible. 
    \end{enumerate}
\end{exercise}

\begin{exercise}[2.5]
    \begin{enumerate}
        \item $\bbP^n$ is a noetherian topological space. 
        \item Every algebraic set in $\bbP^n$ can be written uniquely as a finite
          union of irreducible algebraic sets, no one containing another. These are
          called its \emph{irreducible components}. 
    \end{enumerate}
\end{exercise}

\begin{exercise}[2.6]
    If $Y$ is a projective variety with homogeneous coordinate ring $S(Y)$, show that $\Dim S(Y) = \Dim Y + 1$. 
    [\emph{Hint}: Let $\phi_i: U_i \to \aa^n$ be the homeomorphism of (2.2), 
    let $Y_i$ be the affine variety $\phi_i(Y \cap U_i)$ and let $A(Y_i)$ be its affine coordinate ring.
    Show that $A(Y_i)$ can be identified with the subring of elements of degree 0 of the 
    localized ring $S(Y)x_i$. Then show that $S(Y)_{x_i} \cong A(Y_i)[x_i, x_i^{-1}]$. 
    Now use (1.7), (1.8A), and (Ex 1.10), and look at the transcendence degrees. Conclude also that 
    $\Dim Y = \Dim Y_i$ whenever $Y_i$ is empty.]
\end{exercise}

\begin{exercise}[2.7]
    \begin{enumerate}
        \item $\Dim \bbP^n = n$.
        \item If $Y \subseteq \bbP^n$ is a quasi-projective variety, then $\Dim Y = \Dim \overline{Y}$.
    \end{enumerate}
\end{exercise}

\begin{exercise}[2.8]
    A projective variety $Y \subseteq \bbP^n$ has dimension $n-1$ if
    and only if it is the zero set of a single irreducible homogeneous polynomial
    $f$ of positive degree. $Y$ is called a \emph{hypersurface} in $\bbP^n$.   
\end{exercise}

\begin{exercise}[2.9]
    If $Y \subseteq \aa^n$ is an affine variety, we identify $\aa^n$ with an open
    set $U_0 \subset \pp^n$ by the homeomorphism $\varphi_0$. Then we can speak of
    $\overline{Y}$, the closure of $Y$ in $\pp^n$, which is called the
    \emph{projective closure} of $Y$. 
   \begin{enumerate}
     \item Show that $I(\overline{Y})$ is the ideal generated by $\beta(I(Y))$,
       using the notation of the proof of $(2.2)$. 
     \item Let $Y \subset \aa^n$ be the twisted cubic of \emph{(Ex $1.2$)}. Its
       projective closure $\overline{Y} \subset \pp^n$ is called the \emph{twisted
       cubic curve} in $\pp^3$. Find generators for $I(Y)$ and $I(\overline{Y})$,
       and use this example to show that if $f_1, \ldots, f_r$ generate $I(Y)$,
       then $\beta(f_1), \ldots, \beta(f_r)$ do \emph{not} necessarily generate
       $I(\overline{Y})$. 
   \end{enumerate}
\end{exercise}

\begin{exercise}[2.10]
    Let $Y \subset \bbP^n$ be a nonempty algebraic set, and let $\theta\colon
    \aa^{n+1} \setminus \{(0,\cdots,0\} \to \bbP^n$ be the map which sends the
    point with affine coordinates $(a_0, \cdots, a_n)$ to the point with
    homogeneous coordinates $[a_0 :\cdots: a_n]$. We define the \emph{affine cone}
    over $Y$ to be $$C(Y) = \theta^{-1}(Y) \cup \{(0,\cdots, 0)\}.$$ 
    \begin{enumerate}
        \item Show that $C(Y)$ is an algebraic set in $\aa^{n+1}$, whose ideal is equal to $I(Y)$, considered as an ordinary ideal in $k[x_0, \cdots, x_n]$. 
        \item $C(Y)$ is irreducible if and only if $Y$ is irreducible. 
        \item $\Dim C(Y) = \Dim Y +1$.
    \end{enumerate}
    Sometimes we consider the projective closure $\overline{C(Y)}$ of $C(Y)$ in
    $\bbP^{n+1}$. This is called the \emph{projective cone} over $Y$.
\end{exercise}

\begin{exercise}[2.11]
    A hypersurface defined by a linear polynomial is called a \emph{hyperplane}. 
        \begin{enumerate}
            \item Show that the following two conditions are equivalent for a variety $Y \subset \bbP^n$: 
                \begin{itemize}
                \item[(\emph{i}.)] $I(Y)$ can be generated by linear polynomials
                \item[(\emph{ii}.)] $Y$ can be written as an intersection of hyperplanes. 
                \end{itemize}
                In this case we say that $Y$ is a \emph{linear variety} in $\bbP^n$.
            \item If $Y$ is a linear variety of dimension $r$ in $\bbP^n$, show that $I(Y)$ is minimally generated by $n-r$ linear polynomials 
            \item Let $Y,Z$ be linear varieties in $\bbP^n$, with $\Dim Y = r$, $\Dim Z = s$. If $r+s-n \geq 0$, then $Y \cap Z \neq \emptyset$. 
            Furthermore, if $Y \cap Z \neq \emptyset$, then $Y \cap Z$ is a linear variety 
            of dimension $\geq r+s-n$ (Think of $\aa^{n+1}$ as a vector 
            space over $k$, and work with its subspaces.)
        \end{enumerate}
\end{exercise}

\newpage

\begin{exercise}[2.12]
    For given $n, d>0$, let
    $M_0,\ldots, M_N$ be all the monomials of degree $d$ in the $n+1$ variables
    $x_0, \ldots x_n$, where $N = \binom{n+d}{n} -1.$ We define a mapping
    $\rho_d\colon \bbP^n \to \bbP^N$ by sending the point $P = (a_0, \ldots, a_n)$
    to the point $\rho_d(P) = (M_0(a), \ldots, M_N(a))$ obtained by substituting
    the $a_i$ in the monomials $M_j$. This is called the $d$-uple \emph{embedding}
    of $\bbP^n$ in $\bbP^N$. For example, if $n=1, d=2$, then $N= 2$, and the image
    $Y$ of the $2$-uple embedding of $\bbP^1$ in $\bbP^2$ is a conic. 
    \begin{enumerate}
      \item Let $\theta\colon k[y_0, \ldots, y_N] \to k[x_0, \ldots, x_n]$ be the
        homorphism defined by sending $y_i$ to $M_i$, and let $\mathfrak{a}$ be
        the kernel of $\theta$. Then $\mathfrak{a}$ is homogeneous prime ideal, and
        so $Z(\mathfrak{a})$ is a projective variety in $\bbP^N$. 
      \item Show that the image of $\rho_d$ is exactly $Z(\mathfrak{a})$.
      \item Now show that $\rho_d$ is a homeomorphism of $\bbP^n$ onto the projective
        variety $Z(\mathfrak{a})$. 
      \item Show that the twisted cubic curve in $\bbP^3$
        {\emph{(Ex.\ $2.9$)}} is equal to the $3$-uple
        embedding of $\bbP^1$ in $\bbP^3$, for suitable choice of coordinates. 
    \end{enumerate}
\end{exercise}

\begin{exercise}[2.13]
    Let $Y$ be the image of the $2$-uple embedding of $\bbP^2$ in $\bbP^5$. This is the
    \emph{Veronese surface}. If $Z \subseteq Y$ is a closed curve (a \emph{curve} is 
    a variety of dimension $1$), show that there exists a hypersurface $V \subseteq
    \bbP^5$ such that $V \cap Y = Z$. 
\end{exercise}

\begin{exercise}[2.14]
    Let $\psi\colon \bbP^r \times \bbP^s \to \bbP^N$ be the map defined by sending
    the ordered pair $(a_0, \ldots, a_r) \times (b_0, \ldots, b_s)$ to $(\ldots,
    a_ib_j, \ldots)$ in lexicographic order, where $N = rs + r +s$. Note that
    $\psi$ is well-defined and injective. It is called the \emph{Segre embedding}.
    Show that the image of $\psi$ is a subvariety of $\bbP^N$.
\end{exercise}

\begin{exercise}[2.15]
    Consider the surface $Q$ (a \emph{surface} is variety of dimension $2$) in
    $\bbP^3$ defined by the equation $xy-zw = 0$.
    \begin{enumerate} 
      \item Show that $Q$ is equal to the Segre embedding of $\bbP^1 \times \bbP^1$ in
        $\bbP^3$, for suitable choice of coordinates.
      \item Show that $Q$ contains two families of lines (a \emph{line} is a
        linear variety of dimension $1$) $\{L_t\}, \{M_t\}$, each parametrized by
        $t \in \bbP^1$, with the properties that if $L_t \ne L_u$, then
        $L_t \cap L_u = \emptyset$; if $M_t \ne M_u$, $M_t \cap M_u = \emptyset$,
        and for all $t,u, L_t \cap M_u =$ one point. 
      \item Show that $Q$ contains other curves besides these lines, and deduce that
        the Zariski topology on $Q$ is not homeomorphic via $\psi$ to the product
        topology on $\bbP^1 \times \bbP^1$ (where each $\bbP^1$ has its Zariski
        topology). 
    \end{enumerate}
\end{exercise}

\begin{exercise}[2.16]
    \begin{enumerate}
        \item The intersection of two varieties need not be a variety. For example,
        let $Q_1$ and $Q_2$ be the quadric surfaces in $\bbP^3$ given by the
        equations $x^2-yw = 0$ and $xy - zw =0$, respectively. Show that $Q_1 \cap
        Q_2$ is the union of a twisted cubic curve and a line.
        
        \item Even if the intersection of two varieties is a variety, the ideal of the
        intersection may not be the sum of the ideals. For example, let $C$ be the
        conic in $\bbP^2$ given by the equation $x^2-yz = 0$. Let $L$ be the line
        given by $y = 0$. Show that $C \cap L$ consists of one point $P$, but that
        $I(C) + I(L) \ne I(P)$. 
    \end{enumerate}
\end{exercise}

\begin{exercise}[2.17]
    A variety $Y$ of dimension $r$ in $\bbP^n$ is a \emph{(strict) complete
    intersection} if $I(Y)$ can be generated by $n-r$ elements. $Y$ is a
    \emph{set-theoretic complete intersection} if $Y$ can be written as the
    intersection of $n-r$ hypersurfaces.
    \begin{enumerate}
    \item Let $Y$ be a variety in $\bbP^n$, let $Y = Z(\mathfrak{a})$; and
    suppose that $\mathfrak{a}$ can be generated by $q$ elements. Then show
    that $\Dim Y \ge n-q$. 
    \item Show that a strict complete intersection is a set-theoretic complete
    intersection.
    \item The converse of $(b)$ is false. For example let $Y$ be the twisted
    cubic curve in $\bbP^3$ {\emph{(Ex.\ 2.9)}}. Show that
    $I(Y)$ cannot be generated by two elements. On the other hand, find
    hypersurfaces $H_1,H_2$ of degrees $2,3$ respectively, such that
    $Y = H_1 \cap H_2$. 
    \item It is an unsolved problem whether every closed irreducible curve in
    $\bbP^3$ is a set-theoretic intersection of two surfaces.
    \end{enumerate}
\end{exercise}
