\section{Varities}

\subsection{Affine Varieties}

\begin{exercise}%[1.1]
\begin{enumerate}
\item Let $Y $ be the plane curve $y = x^2 $ (i.e., $Y $ is the zero set of the polynomial $f = y - x^2 $). Show that $A(Y) $ is isomorphic to a polynomial ring in one variable over $k $.
\begin{proof}
	We have that $I(y-x^2) = (y-x^2) $, so $A(Y) = k[x,y] / (y-x^2) $, which is obviously isomorphic to $k[x] $.
\end{proof}
\item Let $Z $ be the plane curve $xy = 1 $. Show that $A(Z) $ is not isomorphic to a polynomial ring in one variable over $k $.
\begin{proof}
	We want to show that $A(Z) = k[x,y] / (xy-1) \not\cong k[x] $.
	Suppose FTSOC that there was an isomorphism $f: A(Z) \to k[x]$.
	Then $f(xy) = f(1) = 1 = f(x)f(y) $ gives us that $f(x) $ is a unit.
	Similarly, we can see that $f(kx) $ are also distinct units.

	As $k $ is a field, $f(k) $ is a field.
	The only subfield of $k[x] $ is $k $, so $f(k) = k $.
	But $f(x) $ is in $k $, so we don't have bijectivity.
\end{proof}
\item Let $f $ be any irreducible quadratic polynomial in $k[x,y] $, and let $W $ be the conic defined by $f $. Show that $A(W) $ is isomorphic to $A(Y) $ or $A(Z) $. Which one is it when?
\begin{proof}
	By Proposition 1.7, $\dim W = \height I(W) \implies $ by Proposition 1.8A $\dim W + \dim k[x,y] / I(W) = 2 $.
	As dimension is non-negative, there are three possibilities for $\dim W $.

	It can't be $2 $ because these correspond to maximal ideals, and maximal ideals here are of the form $(x-a,y-b) $, which isn't principal.

	If it is $1 $, then $\dim k[x,y] / I(W) = 1 $ and thus $A(W) = k[x,y] / I(W) \cong k[f] $ by definition of transcendence degree.
	This corresponds to the $Y $ case.

	If it is $0 $, then $\dim k[x,y] / I(W) = 2 $ and thus $A(W) = k[x,y] / I(W) \cong k[f,g] $ by definition of transcendence degree.
	This corresponds to the $Z $ case, as $k(Z) \cong k[x,x^{-1}] $.
\end{proof}
\end{enumerate}
\end{exercise}

\begin{exercise}%[1.2]
\textit{The Twisted Cubic Curve}. Let $Y \subseteq A^3  $ be the set $Y = \{(t,t^2,t^3 )|t\in k\}   $. Show that $Y $ is an affine variety of dimension 1. Find generators for the ideal $I(Y) $. SHow that $A(Y) $ is isomorphic to a polynomial ring in one variable over $k $. We say that $Y $ is given by the parametric representation $x=t,y=t^2,z=t^3  $.
\end{exercise}
\begin{proof}
	$Y $ is an affine variety because it is closed due to being $Z(y-x^2,z-x^3) $ and irreducible by quotienting out ($k[x,y,z] / I(Y) = k[x] $).
	It is dimension 1 because $\height I(Y) + \dim k[x,y,z] / \mathfrak{p} = \dim k[x,y,z] $ and $1 + 2 = 3 $ by Proposition 1.8A.

	The generators are $y-x^2,z-x^3  $ and $A(Y) \cong k[x] $.
\end{proof}

\begin{exercise}%[1.3]
Let $Y $ be the algebraic set in $\A^3 $ defined by the two polynomials $x^2-yz $ and $xz-x $. Show that $Y $ is a union of three irreducible components. Describe them and find their prime ideals.
\end{exercise}
\begin{proof}
	We can see that $Z(x^2-yz,xz-x) = Z(x^2-yz,x)\cup Z(x^2-yz,z-1)$ because $k $ is an integral domain.
	Then $Z(x^2-yz,x) = Z(yz,x) = Z(y,x)\cup Z(z,x)$, both of which are irreducible by quotienting out.

	Finally, we can see that $Z(x^2-yz,z-1) = Z(x^2-y,z-1) $, which is irreducible by quotienting out.
	Thus $Z(x^2-yz,xz-x) = Z(x,y)\cup Z(x,z)\cup Z(x^2-y,z-1) $.
\end{proof}

\begin{exercise}%[1.4]
Let $Y $ be the algebraic set in $\A^3$ with $\A\times \A $ in the natural way, show that the Zariski topology on $\A^2$ is not the product topology of the Zariski topologies on the two copies of $\A^1 $.
\end{exercise}
\begin{proof}
	In the Zariski topology, we can see that $Z(x^2+y^2-1) $ is closed.
	But it is not closed in the product topology because the Zariski topology on $\A^1 $ is the cofinite topology but the project, so the topology generated by the closed sets of the product topology are finite or finite cross $\A^1 $.
	The product topology doesn't contain spaces that are infinite that don't have $\A^1 $ because finite unions only give finite points in one coordinate and intersecting with a line reduces the line to finite points.
	Clearly $Z(x^2+y^2-1) $ is not the former nor the latter because it is infinite with no line, so it isn't closed in the product topology as the topology is generated by finite unions (which are still finite).
\end{proof}

\begin{exercise}%[1.5]
Show that a $k $-algebra $B $ is isomorphic to the affine coordinate ring of some algebraic set in $\A^n$, for some $n $, if and only if $B $ is a finitely generated $k $-algebra with no nilpotent elements.
\end{exercise}
\begin{proof}
	$\implies) $ By the Nullstellensatz, all affine coordinate rings have no nilpotent elements ($k[x_{1},\cdots,x_n] / I(Z(J)) $ because an affine coordinate ring is polynomial ring quotiented by the ideal of an affine algebraic set (and hence is the zeros of some ideal $J $ by closure)).
	They are finitely generated because they are Noetherian.

	$\Leftarrow) $ If $B $ is a finitely generated $k $-algebra with no nilpotents: 
	Let the generators be $x_{1},\cdots, x_n $.
\end{proof}

\begin{exercise}%[1.6]
Any nonempty open subset of an irreducible topological space is dense and irreducible. If $Y $ is a subset of a topological space $X $, which is irreducible in its induced topology, then the closure $\overline{Y}  $ is also irreducible.
\end{exercise}
\begin{proof}
	Let $Y $ be an open subset of an irreducible topological space $X $.
	Then $\overline{Y}  $ and $Y^C $ contain $X $ and are closed, so either $\overline{Y} = X  $ or $Y^C = X $.
	As $Y $ is non-empty, it must be the former case and thus $\overline{Y} = X  $, giving us density.

	Suppose FTSOC that $Y $ is a union of two closed sets in $Y $, $C_{1},C_{2} $.
	As they are closed in the induced topology, $\exists C_{1}',C_{2}' $ closed in $X $ s.t. $C_{1} = Y \cap C_{1}' $ and $C_{2} = Y\cap C_{2}' $.
	Thus $Y = C_{1} \cup C_{2} = (Y\cap C_{1}') \cup (Y \cap C_{2}') = Y \cap (C_{1}'\cup C_{2}') \implies C_{1}'\cup C_{2}' \supseteq Y$.
	As $Y $ is open, $Y^C $ is closed.
	Hence $X = Y^C \cup C_{1}'\cup C_{2}' $, and none of these are the whole space by conditions of definitions and hypotheses.

	Second part:
	Suppose FTSOC that $\overline{Y}  $ isn't irreducible.
	Then there are $C_{1},C_{2} $ closed, non-empty in $\overline{Y}  $ s.t. $\overline{Y} = C_{1}\cup C_{2} $.
	Then by definition, $C_{1}\cap Y $ and $C_{2}\cap Y $ are closed in $Y $, which then leads to a closed cover of $Y $.
	They are non-empty because if, WLOG, $C_{2}\cap Y = \emptyset $, then $Y\subseteq C_{1} $, giving us a strictly smaller closed set that contains $Y $ (the strict is because $C_{2} $ is non-empty).
\end{proof}

\begin{exercise}%[1.7]
~
\begin{enumerate}
\item Show that the following conditions are equivalent for a topological space $X$: $(i)$ $X$ is noetherian; $(ii)$ every nonempty family of closed subsets has a minimal element; $(iii)$ $X$ satisfies the ascending chain condition for open subsets; $(iv)$ every nonempty family of open subsets has a maximal element.
\begin{proof}
	$(i)\implies (ii):$ Suppose that there was no minimal element.
	Then there is an infinitely, strictly decreasing chain of closed sets, a contradiction of Noetherianness.

	$(ii) \implies (iii) $: Take an ascending chain of open sets.
	The complements lead to a descending chain of closed sets, which stabilizes.
	The minimal element of this chain's complement is the maximal element of the chain.

	$(iii)\implies (iv) $: Basically the same argument as in $(i)\implies (ii) $.

	$(iv)\implies (1) $: Take a descending chain of closed sets.
	The complements lead to an ascending chain of open sets, which have a maximal element as the chain is a family.
	Hence it stabilizes, which implies that the descending chain stabilizes.
\end{proof}
\item A noetherian topological space is \emph{quasi-compact,} i.e., every open cover has a finite subcover.
\begin{proof}
	Take an arbitrary $U_{1} $ in the cover.
	Then take a $U_{2} $ that contains a point not in $U_{1} $, $U_{3} $ that contains a point not in $U_{1}\cup U_{2} $, and construct inductively.
	Then we have an ascending chain
	\[
		U_{1} \subseteq \bigcup_{i=1}^2 U_i \subseteq \bigcup_{i=1}^3 U_i \subseteq \cdots
	.\] 
	This stabilizes after finite steps by Noetherian-ness, and it has to stabilize at the whole space by construction.
	This gives us a finite subcover.
\end{proof}
\item Any subset of a noetherian topological space is noetherian in its induced topology.
\begin{proof}
	Let $X\subseteq Y $, $Y $ a Noetherian space.
	Take an ascending chain of open sets in the subset
	\[
		U_{1} \subseteq U_{2} \subseteq \cdots
	.\]
	Each $U_{i} = U_i' \cap X$ with $U_i' $ open in $Y $.
	The chain $U_{1}' \subseteq U_{2}' \subseteq \cdots $ stabilizes by Noetherian-ness, so the intersections with $X $ stabilize.
	Hence the chain stabilizes.
\end{proof}
\item A noetherian space which is also Hausdorff must be a finite set with the discrete topology.    
\begin{proof}
	Because noetherian spaces are quasi-compact, all compact subspaces are closed in a Hausdorff space, and all subsets are noetherian and hence quasi-compact, all subsets are closed.
\end{proof}
\end{enumerate}
\end{exercise}

\begin{exercise}%[1.8]
Let $Y$ be an affine variety of dimension $r$ in $\A^n$. Let $H$ be a hypersurface in $\A^n$, and assume $Y \not\subseteq H$. Then every irreducible component of $Y \cap H$ has dimension $r-1$. (See $(7.1)$ for a generalization.)
\end{exercise}
\begin{proof}
	Take a maximal chain of $Y$
	\[
		Y_{0}\subseteq Y_{1}\subseteq \cdots \subseteq Y_r
	.\] 
	Because $Y $ is an affine variety, $Y_r = Y $ and because this is in $\A^n $, $Y_{0} $ is a point.
	Take an irreducible compoment $D $ of $Y\cap H $

	Suppose there was an irreducible component of $Y \cap H $ with dimension less than $r-1 $, call it $D $.
\end{proof}

\begin{exercise}%[1.9]
Let $\mathfrak{a} \subseteq A = k[x_1,\ldots,x_n]$ be an ideal which can be generated by $r$ elements. Then every irreducible component of $Z(\mathfrak{a})$ has dimension $\ge n-r$.
\end{exercise}
\begin{proof}
	Suppose FTSOC that we had an irreducible compoment of $Z(\mathfrak{a}) $ with dimension $< n-r $.
	Then we have a chain 
\end{proof}

\begin{exercise}%[1.10]
~
\begin{enumerate}
\item If $Y$ is any subset of a topological space $X$ then $\dim Y \leq \dim X$. 
\begin{proof}
	Easy. Any ascending chain of distinct irreducible closed subsets of $Y $ are lead to distinct irreducible and closed in $X$ (use Exercise 1.1.6) by definition of being closed in the subspace topology.
	The dimension of $X $ is the supremum of such chains so it is $\ge \dim Y $.
\end{proof}
\item If $X$ is a topological space which is covered by a family of open subsets $\{U_{i}\}$, then $\dim X = \sup \dim U_i$.
\begin{proof}
	Let $X_{0}\subsetneq X_{1}\subsetneq \cdots \subsetneq X_n $ be the maximal chain of distinct, closed, irreducible subsets of $X $.
\end{proof}
\item Give an example of a topological space $X$ and a dense open subset $U$ with $\dim U < \dim X$. 
\begin{proof}
	Consider $[0,1] $ with the Euclidean metric topology.
	Take $\Q\cap [0,1] $.
\end{proof}
\item If Y is a closed subset of an irreducible finite-dimensional topological space $X$, and if $\dim Y = \dim X$, then $Y= X$. 
\begin{proof}
\end{proof}
\item Give an example of a noetherian topological space of infinite dimension.  
\begin{proof}
	Take $\A^\infty $, which corresponds to $k[x_{1},x_{2},\cdots] $.
	This ring is Noetherian, as any ideal belongs to a finite transcendence degree polynomial ring, which are Noetherian, hence the space is Noetherian.
	It is infinite dimensional because the ring is infinite dimensional.
\end{proof}
\end{enumerate}
\end{exercise}

\begin{exercise}%[1.11]
Let $Y \subseteq \A^3$ be the curve given parametrically by $x = t^3$, $y= t^4$, $z = t^5$. Show that $I(Y)$ is a prime ideal of height $2$ in $k[x,y,z]$ which cannot be generated by $2$ elements. We say $Y$ is \emph{not a local complete intersection}---cf.~\emph{(Ex.~$2.17$)}. 
\end{exercise}

\begin{exercise}%[1.12]
Give an example of an irreducible polynomial $f \in \mathbf{R}[x,y]$, whose zero set $Z(f)$ in $\A_{\mathbf{R}}^2$ is not irreducible
\emph{(cf.~$1.4.2$)}.
\end{exercise}
\begin{proof}
	An example is $x^2+y^2+1 $.
	This is irreducible because it has no real roots and all factors would have degree 1.
	The zero set is the empty set, which isn't irreducible.
\end{proof}
